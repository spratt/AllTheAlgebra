\chapter{Rubik's Cube Math}

\section{Terminology}

First we name the faces of the cube, these are arbitrary but
important.  We choose a front face $F$, and an up face $U$ and this
uniquely determines the rest of the faces: back $B$, down $D$, left
$L$, and right $R$.

\begin{math}
  %
  F = \text{white} \\
  U = \text{red} \\
  %
\end{math}

Which on my cube gives:

\begin{math}
  %
  B = \text{yellow} \\
  D = \text{orange} \\
  L = \text{blue} \\
  R = \text{green} \\
  %
\end{math}

We refer to the little cubes that make up the whole cube as
``cubies''.  Each cubie has a unique name determined by its colors.
An edge cubie has two colors, and an example of one is the front-up
cubie (which in my case is colored white and red), or ``fu''.  A
corner cubie has three colors, e.g. fur (white, red, green).  

A center cubie only has one color, but since they always stay in the
same place relative to other center cubies, we can disregard them
except to determine which face is which in an unsolved cube.

Using cycle notation, we can now define operations on the cubes.  The
move $F$ is to turn the front face clockwise by 90 degrees.  Moving
the front face anti-clockwise by 90 degrees is the inverse of the $F$
operation, so we consider this $F^{-1}$.  Moving the front face
clockwise or anti-clockwise by 180 degrees is either $F^2$ or
$F^{-2}$, whichever is preferred.

\begin{math}
  %
  F = \text{(ful fur fdr fdl) (fu fr fd fl)} \\
  U = \text{(ful bul bur fur) (fu lu bu ru)} \\
  B = \text{(bur bul bdl bdr) (bu bl bd br)} \\
  D = \text{(fdl fdr bdr bdl) (fd rd bd ld)} \\
  L = \text{(bul ful fdl bdl) (lu fl dl bl)} \\
  R = \text{(fur bur bdr fdr) (ru br rd fr)} \\
  %
\end{math}

\section{Composition}

We can define macro operations by compositions of these cyclic
functions.  For example, the conjugacy of $D$ by $F$ is:

\begin{displaymath}
    FDF^{-1}
  = \text{(ful fur fdr fdl) (fu fr fd fl)     % F
          (fdl fdr bdr bdl) (ld bd rd fd)     % D
          (fdl fdr fur ful) (fl fd fr fu)}    % F^{-1}
\end{displaymath}

But we can consider the edge cubies separately from the corner cubies,
since they are disjoint.

\begin{align*}
  %
  FDF^{-1} &= \text{AB} \\
  &       \\
  A
  &= \text{(ful fur fdr fdl)(fdl fdr bdr bdl)(fdl fdr fur ful)} \\
  &= \text{(ful fdl bdr bdl)(fur)(fdr)} \\
  &= \text{(ful fdl bdr bdl)} \\
  & \\
  B
  &= \text{(fu fr fd fl)(ld bd rd fd)(fl fd fr fu)} \\
  &= \text{(fl ld bd rd)(fd)(fr)(fu)} \\
  &= \text{(fl ld bd rd)} \\
  & \\
  FDF^{-1}
  &= \text{AB} \\
  &= \text{(ful fdl bdr bdl)(fl ld bd rd)}
  %
\end{align*}

Notice that this move is disjoint to $U$ except in one corner, ful.

Let $M_1 = FDF^{-1}$.  We create the commutator of $M_1$ and $U$ as
follows:

\begin{align*}
  %
  T_1
  &= U^{-1}M_1^{-1}UM_1 \\
  &= CD \\
  & \\
  C
  &= \text{(fur bur bul ful)(bdl bdr fdl ful)(ful bul bur fur)(ful fdl bdr bdl)} \\
  &= \text{(bdl ful fur)(fdl)(bdr)(bul)(bur)} \\
  &= \text{(bdl ful fur)} \\
  & \\
  D
  &= \text{(ru bu lu fu)(rd bd ld fl)(fu lu bu ru)(fl ld bd rd)} \\
  &= \text{(ru)(bu)(lu)(fu)(rd)(bd)(ld)(fl)} \\
  & \\
  T_1
  &= CD = \text{(bdl ful fur)(ru)(bu)(lu)(fu)(rd)(bd)(ld)(fl)} \\
  &= \text{(bdl ful fur)} \\
  %
\end{align*}
  
As it turns out, this is the minimal move on corner cubies.  Using
this, we can create moves on any three corner cubies by conjugation.
Let's say we would like to move bul instead of bdl.  We can do this
using conjugation.  First we find a move which moves the cubie we want
to move into the cubicle we know how to move.  In the example of
moving bul of bdl, we need only move bul bdl, so our move $Z_1$ is
simply $B$.  Then we conjugate, ignoring the edge cubies since they
will be moved back into position.

\begin{align*}
  %
  Z_1^{-1}T_1Z_1
  &= B^{-1}T_1B \\
  &= \text{(bdr bdl bul bur)(bdl ful fur)(bur bul bdl bdr)} \\
  &= \text{(bul ful fur)}
  %
\end{align*}

So our full move is (read right to left):

\begin{align*}
  %
  Z_1^{-1} T_1 Z_1
  &= B^{-1} T_1 B \\
  &= B^{-1} U^{-1} M_1^{-1} U M_1 B \\
  &= B^{-1} U^{-1} F^{-1}D^{-1}F U FDF^{-1} B
  %
\end{align*}

Now let's say we would like (bdl bur bul).  We need to move ful to
bur, and fur to bul.  This can be done easily with $Z_2 = U^2$, which
is conveniently its own inverse.

\begin{align*}
  %
  U^2
  &= UU = \text{(ful bul bur fur)(ful bul bur fur)} \\
  &= \text{(ful bur)(bul fur)} \\
  & \\
  Z_2^{-1}T_1Z_2
  &= U^2T_1U^2 = \text{(ful bur)(bul fur)(bdl ful fur)(ful bur)(bul fur)} \\
  &= \text{(bul bdl bur)}
  %
\end{align*}

So our full move is (read right to left):

\begin{align*}
  %
  Z_2^{-1} T_1 Z_2
  &= U^2 T_1 U^2 \\
  &= U^2 U^{-1} M_1^{-1} U M_1 U^2 \\
  &= U F^{-1}D^{-1}F U FDF^{-1} U^2
  %
\end{align*}
  
