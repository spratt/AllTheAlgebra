\chapter{Rubik's Cube Math}

First we name the faces of the cube, these are arbitrary but
important.  We choose a front face $F$, and an up face $U$ and this
uniquely determines the rest of the faces: back $B$, down $D$, left
$L$, and right $R$.

\begin{math}
  %
  F = \text{white} \\
  U = \text{red} \\
  %
\end{math}

Which on my cube gives:

\begin{math}
  %
  B = \text{yellow} \\
  D = \text{orange} \\
  L = \text{blue} \\
  R = \text{green} \\
  %
\end{math}

We refer to the little cubes that make up the whole cube as
``cubies''.  Each cubie has a unique name determined by its colors.
An edge cubie has two colors, and an example of one is the front-up
cubie (which in my case is colored white and red), or ``fu''.  A
corner cubie has three colors, e.g. fur (white, red, green).  

A center cubie only has one color, but since they always stay in the
same place relative to other center cubies, we can disregard them
except to determine which face is which in an unsolved cube.

Using cycle notation, we can now define operations on the cubes.  The
move $F$ is to turn the front face clockwise by 90 degrees.  Moving
the front face anti-clockwise by 90 degrees is the inverse of the $F$
operation, so we consider this $F^{-1}$.  Moving the front face
clockwise or anti-clockwise by 180 degrees is either $F^2$ or
$F^{-2}$, whichever is preferred.

\begin{math}
  %
  F = \text{(ful fur fdr fdl) (fu fr fd fl)} \\
  U = \text{(ful bul bur fur) (fu lu bu ru)} \\
  B = \text{(bur bul bdl bdr) (bu bl bd br)} \\
  D = \text{(fdl fdr bdr bdl) (fd rd bd ld)} \\
  L = \text{(bul ful fdl bdl) (lu fl dl bl)} \\
  R = \text{(fur bur bdr fdr) (ru br rd fr)} \\
  %
\end{math}

We can define macro operations by compositions of these cyclic
functions.  For example, the conjugacy of $D$ by $F$ is:

\begin{align*}
  %
  F^{-1}DF 
  &= \text{(fdl fdr fur ful) (fl fd fr fu)     % F^{-1}
           (fdl fdr bdr bdl) (fd rd bd ld)     % D
           (ful fur fdr fdl) (fu fr fd fl)} \\ % F
  &= \text{(fur bdr bdl fdr) (ful) (fdl) (fr rd bd ld) (fu) (fl)} \\
  &= \text{(fur bdr bdl fdr) (fr rd bd ld)}
  %
\end{align*}
  
And the commutator of $F$ and $D$ is:

\begin{align*}
  %
  F^{-1}D^{-1}FD
  = &\text{(fdl fdr fur ful) (fl fd fr fu)     % F^{-1}
           (bdl bdr fdr fdl) (ld bd rd fd)} \\ % D^{-1}
    &\text{(ful fur fdr fdl) (fu fr fd fl)     % F
           (fdl fdr bdr bdl) (fd rd bd ld)} \\ % D
  = &\text{(fdl bdl) (fdr fur) (bdr) (ful) (fd fr ld) (rd) (bd) (fu) (fl)} \\
  = &\text{(fdl bdl) (fdr fur) (fd fr ld)}
  %
\end{align*}
  
Remembering that $(F^{-1}D^{-1}FD)^{-1} = D^{-1}F^{-1}DF$.

And since 2 cycles are their own inverses, we can run this move twice to attain:

\begin{align*}
  %
  M = F^{-1}D^{-1}FDF^{-1}D^{-1}FD 
  = &\text{(fdl bdl) (fdr fur) (fd fr ld) (fdl bdl) (fdr fur) (fd fr ld)} \\
  = &\text{(fd ld fr)}
  %
\end{align*}

We now have a minimal move on the edge cubies.  For those following
along, you may notice that the corner cubies are in the correct
location but twisted, which our notation does not capture.


