\chapter{Rubik's Cube Math}

First we name the faces of the cube, these are arbitrary but
important.  We choose a front face $F$, and an up face $U$ and this
uniquely determines the rest of the faces: back $B$, down $D$, left
$L$, and right $R$.

\begin{math}
  %
  F = \text{white} \\
  U = \text{red} \\
  %
\end{math}

Which on my cube gives:

\begin{math}
  %
  B = \text{yellow} \\
  D = \text{orange} \\
  L = \text{blue} \\
  R = \text{green} \\
  %
\end{math}

We refer to the little cubes that make up the whole cube as
``cubies''.  Each cubie has a unique name determined by its colors.
An edge cubie has two colors, and an example of one is the front-up
cubie (which in my case is colored white and red), or ``fu''.  A
corner cubie has three colors, e.g. fur (white, red, green).  

A center cubie only has one color, but since they always stay in the
same place relative to other center cubies, we can disregard them
except to determine which face is which in an unsolved cube.

Using cycle notation, we can now define operations on the cubes.  The
move $F$ is to turn the front face clockwise by 90 degrees.  Moving
the front face anti-clockwise by 90 degrees is the inverse of the $F$
operation, so we consider this $F^{-1}$.  Moving the front face
clockwise or anti-clockwise by 180 degrees is either $F^2$ or
$F^{-2}$, whichever is preferred.

\begin{math}
  %
  F = (ful fur fdr fdl) (fu fr fd fl) \\
  U = (ful fur bur bul) (fu ru bu lu) \\
  B = (bul bur bdr bdl) (bu bl bd br) \\
  D = (fdl fdr bdr bdl) (fd rd bd ld) \\
  L = (bul ful fdl bdl) (lu fl dl bl) \\
  R = (bur fur fdr bdr) (ru fr dr br) \\
  %
\end{math}
  

