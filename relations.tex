\chapter{Relations}

\section{Mappings}

A mapping $f : A \rightarrow B$ is said to be \emph{onto} or
\emph{surjective} if $ f(A) = B $.

A mapping (as defined as above) is said to be \emph{one-to-one} or
\emph{injective} if $ f(a) = f(b) \implies a = b $.

A mapping is said to be \emph{bijective} if it is both surjective and
injective.

\section{Equivalence Relation}

A relation $ R = \{ (a,b) | a,b \in A \} $ is said to be
\emph{reflexive} if $ \forall a \in A: (a,a) \in R $.

A relation (as defined above) is said to be \emph{symmetric} if $
(a,b) \in R \implies (b,a) \in R $.

A relation is said to be \emph{transitive} if $ (a,b) \in R and (b,c)
\in R \implies (a,c) \in R $.

A relation is said to be an \emph{equivalence relation} if it is
reflexive, symmetric and transitive.

