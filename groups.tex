\chapter{Groups}

A \emph{group} is a monoid where every element has a full inverse.

The \emph{order} of a group is the number of elements in the group, or
infinity if the group is infinite.

A \emph{subgroup} of a group $G$ is a subset of the elements in $G$
that form a group.  If $H$ is a subgroup of $G$, we say $H < G$.

\section{Cyclic Groups}

A \emph{cyclic group} is a group generated by the operation being
repeatedly performed on a generator element.  In the case of a
multiplicative group: $<a> = \{ a^k | k \in \zz \}$, in which case $a$
is the generator element.

The order of a cyclic group is the smallest positive integer $k$ such
that $a^k$ is the identity.

If $|<a>| = n$, the subgroup $<a^k> = <a^d>$ where $d = gcd(k,n)$ and
$|<a^k>| = \frac{n}{gcd(k,n)}$.

\section{Homomorphisms}

A \emph{homomorphism} is a map $\phi : S \rightarrow S'$ where $G =
(\dot,S,e)$ and $G' = (\otimes,S',e')$ such that $\phi(g) \otimes
\phi(h) = \phi(g \dot h)$, where $g,h \in S$.

A homomorphism $S \rightarrow S$ is an \emph{endomorphism}.

A homomorphism where $\phi$ is surjective (onto) is called an
\emph{epimorphism}.

A homomorphism where $\phi$ is bijective (onto and one-to-one) is
called an \emph{isomorphism}.
