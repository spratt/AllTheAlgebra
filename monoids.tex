\chapter{Monoids}

A \emph{binary operation} is a mapping from $A \cross B$ to $ C $.

A binary operation is said to be \emph{closed} if it maps from $A
\cross A$ to $A$.

A closed binary operation ( $ \dot $ ) is said to be
\emph{associative} if $ \forall a,b,c \in A: ( a \dot b ) \dot c = a
\dot ( b \dot c )$.

A \emph{monoid} is a 3-tuple $(\dot, M, e)$ where $\dot$ is an
associative, closed binary operation on $M$ and the identity element $
e \in M: \forall m \in M: e \dot m = m $.

An \emph{inverse} is an element $\forall a \in A \exists a' \in A: a
\dot a' = e \text{OR} a' \dot a = e$.  This element is a \emph{left
  inverse} if $a' \dot a = e$ or a \emph{right inverse} if $a \dot a'
= e$ and is a \emph{full inverse} if it is both a right inverse and a
left inverse.

It is easy to prove that a left inverse is unique given right
cancelation, a right inverse is unique given left cancelation and a
full inverse is unique from either left or right cancelation.

\section{Row Monomials}

We define the \emph{row monomials}, $RM(n)$, as the set of all $0,1$
matrices with exactly one $1$ per row.  We will show that
$(\dot,RM(n),I_n)$ forms a monoid, where $\dot$ is matrix
multiplication, and $I_n$ is the identity for $n \cross n$ matrices.
Since matrix multiplication is associative in general, we need only
prove the following claim:

\begin{claim}

  $RM(n)$ is closed under matrix multiplication.

\end{claim}

\begin{proof}

  Let $A,B \in RM(n)$.  $A$ is composed of the elements $a_{ij}$.  For
  all $1 \leq i \leq n$ there exists a unique $l$ such that $a_{il} =
  1$ and for all $j \neq l$, $a_{ij} = 0$.

  \[
  (AB)_{ij} = \summ{k=1}{n} a_{ik}b_{kj} = a_{il}b_{lj} = b_{lj}
  \]

  \[
  \therefore (AB)_{ij} \in RM(n)
  \]
  
\end{proof}

Thus proving $RM(n)$ is a monoid.

\section{Deterministic Finite Automata}

A \emph{Deterministic Finite Automata} (DFA) is a 5-tuple
$(A,S,s,F,\delta)$ where $A$ is a set of symbols called the
\emph{alphabet}, $S = \{ s_1, s_2, ..., n \} $ is the set of states,
$s \in S$ is the \emph{start state}, $F \subseteq S$ is the set of all
accept states, and finally $\delta: A \cross S \rightarrow S$ is the
transition function which returns the next state when given the
current state and a symbol to read.  A DFA reads a string of
characters from the alphabet one character at a time, transitioning
from state to state until no more characters can be read.  At which
point, the DFA either \emph{accepts} the string if the current state
is in the set $F$, or \emph{rejects} the string otherwise.

We can model a DFA using row monomials thusly:

\begin{enumerate}
\item to each state $s_k \in S : 1 \leq k \leq n$, we associate a vector
  such that for $1 \leq i \leq n$:
  \begin{math}
    [s]_i = \left\{
      \begin{array}{l l}
        1 & \text{if } i = k \\
        0 & \text{otherwise}
      \end{array} \right.
  \end{math}

\item to each character $a \in A$, we associate a matrix from $RM(n)$,
  such that for $1 \leq i \leq n$ and $1 \leq j \leq n$:
  \begin{math}
    [a]_{ij} = \left\{
      \begin{array}{l l}
        1 & \text{if } \delta(s_i,a) = s_j \\
        0 & \text{otherwise}
      \end{array} \right.
  \end{math}
\end{enumerate}