\chapter{Division}

\section{The Extended Euclidean Algorithm}

Given numbers $a,b$ with which we would like to calculate $gcd(a,b)$,
we can do so using the Euclidean Algorithm.  Without loss of
generality, say $ a > b $.  Let $r_0 = a, r_1 = b$.  We divide
$r_{i-2},r_{i-1}$ to give us $r_{i-2} = r_{i-1} \times q_i + r_i$.  We
continue dividing until we find the smallest $k$ such that $r_k = 0$, at which
point $gcd(a,b) = r_{k-1}$.

The ``extension'' of the Euclidean Algorithm also gives us $s_i, t_i$
such that $ r_i = b \times s_i + a \times t_i $.  This is
important for \emph{RSA Encryption}.  We define $s_0 = 1, s_1 = 0, t_0
= 0, t_1 = 1$ and calculate $ s_i = s_{i-2} - q_is_{i-1} $ and $ t_i =
t_{i-2} - q_it_{i-1} $.

We build a table thusly:

\[
\begin{array}{ c | c c c c }
  i & q_i & r_i & s_i & t_i \\
  \hline
  0 & n/a & 39  & 1   & 0   \\
  1 & n/a & 14  & 0   & 1   \\
  2 & 2   & 11  & 1   & -2  \\
  3 & 1   & 3   & -3  & 7   \\
  4 & 3   & 2   & 7   & -16 \\
  5 & 1   & 1   & -10 & 23  \\
  6 & 1   & 0   & \leftarrow & \text{done}
\end{array}
\]  

So we know $gcd(14,39) = 1 = 39 \times -10 + 14 \times 23$.